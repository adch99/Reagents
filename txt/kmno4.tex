Here we look at series of reactions using \ce{KMnO_4} in acid medium.

% \ce{} is a bit flexible in whether you write KMnO_4 (the correct way)
% or KMnO4 (the incorrect but workable way).

$ \ce{KMnO4 + H^+ -> Mn^2+ + K^+ + H_2O} $

$ \ce{KMnO4 + C=C -> 2CO2} $

We can also see the following aligned equations:

\begin{align}
	\ce{Mn^{7+} + 5e- &<=> Mn2+} \\
	\ce{Mn^{6+} + 5e- &<=> Mn2+}
\end{align}

As you can see, the align environment makes it look better.
We can also link to any section i.e reagent or image.

For an allied oxidant see \hyperref[k2cr2o7]{\ce{K2Cr2O7}}. 
Look here for the \hyperref[fig:k2cr2o7]{structure of \ce{K2Cr2O7}}.

The rest of this document is basically placeholder text and
images to show you how it will look with a populated list.